\documentclass[a4paper]{article}

\usepackage[english]{babel}
\usepackage[OT1,T1]{fontenc}
\usepackage{amsmath}


\begin{document}
\selectlanguage{english}


In Natural units, the following holds:

\begin{equation}
    E = \sqrt{m^2 + p^2} \to m^2 = E^2 - p^2 \label{eq:mass2}
\end{equation}

If $\beta_i = \frac{p_i}{E}$, then $\vec{\beta}=\frac{\vec{p}}{E}$ and
$\beta^2 = \frac{p^2}{E^2}$, which can be put into the following:
\begin{eqnarray}
    E &=& m \gamma = \frac{m}{\sqrt{1 - \beta^2}} =
            \frac{m}{\sqrt{1 - \frac{p^2}{E^2}}} =
            \frac{m E}{\sqrt{E^2 - p^2}} \\
        &\overset{\text{Eq.~(\ref{eq:mass2})}}{=}&
            \frac{m E}{m} = E \quad\quad \text{\hfill Q.E.D.} 
\end{eqnarray}

\textbf{Alternatively}

In Natural units, $\vec{\beta} = \vec{v}$ then
\begin{equation}
    \frac{\vec{p}}{E} = \frac{m \gamma\vec{v}}{m \gamma} = \vec{\beta}.
\end{equation}



\end{document}
