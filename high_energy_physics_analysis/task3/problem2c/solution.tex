\documentclass[a4paper]{article}

\usepackage[english]{babel}
\usepackage[OT1,T1]{fontenc}
\usepackage{amsmath}


\begin{document}
\selectlanguage{english}


Since ${\rm p-value}$ of $H_0$ is $\sim 0.007$ ($\sim 2.5\sigma$), while ${\rm 
p-value}$ of $H_A$ is $\sim 0.09$ ($\sim 1.4\sigma$), $H_0$ is less likely to be 
true. Also, $\rm CL_S$ is $\sim 0.006$, which is well below a typical exclusion 
value of 0.05. Based on these observations we could exclude $H_0$ at $99.3\%$ 
level in favor of $H_A$.

Note that $H_A$ does not stand great as well, it alone could be excluded at
$91\%$ level. However, it falls short below a loose exclusion
requirement of $95\%$. In physics, we use $5\sigma$ (${\rm p-value} \sim 2.8 
\cdot 10^{-7}$) rule.



\end{document}
