\documentclass[a4paper]{article}

\usepackage[english]{babel}
\usepackage[OT1,T1]{fontenc}
\usepackage{amsmath}


\begin{document}
\selectlanguage{english}


Since
% 
\begin{equation}
    {\rm p-value} = P(x_{\rm obs} \ge X),
\end{equation}
for $n$-dim case 
\begin{equation}
    {\rm p-value}_n = P(\vec{x}_{\rm obs} \ge \vec{X}).
\end{equation}
% 
In the case of the 
mutually independent (probability distributions) dimensions,
% 
\begin{equation}
    {\rm p-value}_n = \prod_i^n P(x_{i,\rm obs} \ge X_i)
\end{equation}
% 
For $n$-dim Uniform distribution:
% 
\begin{equation}
  {\rm p-value}_n =
    \begin{cases}
      \left(\frac{x}{b - a} \right)^n & \text{if } a < x \le \frac{b - a}{2}\\
      \left(\frac{b - x}{b - a} \right)^n & \text{if } b > x > \frac{b - a}{2} \\
      0 & \text{otherwise}
    \end{cases}
\end{equation}
% 
For ${\rm Uniform}(0, 10)$ and $\vec{x}_{\rm obs} = \{6...6\}$, $n = 50$:
\begin{equation}
    {\rm p-value}_{50} = \left(\frac{10 - 6}{10} \right)^{50} \approx 1.27 \cdot 
10^{-20}.
\end{equation}
% 
Comparing to $1/2^{50} \approx 8.88 \cdot 10^{-16}$, ${\rm p-value}_{50}$ \
is over $10^4$ times less likely. Note that ex. 1c gives a result of about
${\rm p-value} \approx 7 \cdot 10^{-3}$, which is less discriminating due to 
a collective averaged (less precise) behavior. In this case, exact 
multidimensional information is used to discriminate.


\end{document}
